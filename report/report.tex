\documentclass[10pt,a4paper]{article}
\usepackage[utf8]{inputenc}
\usepackage[francais]{babel}
\usepackage[T1]{fontenc}
\usepackage{amsmath}
\usepackage{amsfonts}
\usepackage{amssymb}
\usepackage{graphicx}
\usepackage{listings}
\usepackage{titling}
\usepackage{lstautogobble}
\author{\textsc{Waelkens} Dimitri, \textsc{Lempereur} Martin}
\pretitle{%
  \begin{center}
  Département d'Informatique \\
        Université de Mons \\[2ex]
  \includegraphics[height=4ex]{UMONS}\hspace{2em}
         \raisebox{-1ex}{\includegraphics[height=6ex]{UMONS_FS}}\\[\bigskipamount]
 
    
}
\posttitle{\end{center}}

\begin{document}
\title{\LARGE{Devoir de compilation : Rapport}}
\date{\today}
\maketitle
\newpage
\tableofcontents
\newpage

\section{Consigne}
Il nous était demandé dans le cadre du cours de compilation de réaliser un moteur de templating simplifié.
Ce moteur doit recevoir un fichier de données ainsi qu'un ficher de template qui sera utilisé pour venir y insérer ces données.
Ce genre de moteur de templating permet à partir d'un même modèle d'afficher plusieurs fichiers avec des données différentes sans devoir pour autant tout réécrire.
\section{Grammaire}
Nous avons d'abord implémenté la grammaire décrite dans les consignes.
Ensuite, nous avons inséré plusieurs règles pour gérer les opérations sur les entiers ainsi que certaines expressions booléennes.
La grammaire résultante est décrite ci-dessous :\\

\begin{flushleft}
Rule 0 : $S' \rightarrow prog$\\
Rule 1 : $prog \rightarrow TXT$\\
Rule 2 : $prog \rightarrow dumbo\_block$\\
Rule 3 : $prog \rightarrow TXT\ prog$\\
Rule 4 : $prog \rightarrow dumbo\_block\ prog$\\
Rule 5 : $dumbo\_block \rightarrow BEGIN\ expression\_list\ END$\\
Rule 6 : $expression\_list \rightarrow\ expression\ ;$\\
Rule 7 : $expression\_list \rightarrow expression\ ;\ expression\_list$\\
Rule 8 : $expression \rightarrow PRINT\ string\_expression$\\
Rule 9 : $expression \rightarrow PRINT\ int\_expression$\\
Rule 10 : $string\_expression \rightarrow STRING$\\
Rule 11 : $string\_expression \rightarrow string\_expression\ .\ string\_expression$\\
Rule 12 : $expression \rightarrow ID\ ASSIGN\ string\_expression$\\
Rule 13 : $string\_expression \rightarrow ID$\\
Rule 14 : $string\_list \rightarrow (\ string\_list\_interior\ )$\\
Rule 15 : $string\_list\_interior \rightarrow STRING\ ,\ string\_list\_interior$\\
Rule 16 : $string\_list\_interior \rightarrow STRING$\\
Rule 17 : $expression \rightarrow ID\ ASSIGN\ string\_list$\\
Rule 18 : $expression \rightarrow FOR\ ID\ IN\ string\_list\ DO\ expression\_list\ ENDFOR$\\
Rule 19 : $expression \rightarrow FOR\ ID\ IN\ ID\ DO\ expression\_list\ ENDFOR$\\
Rule 20 : $int\_expression \rightarrow ID$\\
Rule 21 : $int\_expression \rightarrow NBR$\\
Rule 22 : $expression \rightarrow ID\ ASSIGN\ int\_expression$\\
Rule 23 : $int\_expression \rightarrow int\_expression\ -\ int\_expression$\\
Rule 24 : $int\_expression \rightarrow int\_expression\ +\ int\_expression$\\
Rule 25 : $int\_expression \rightarrow int\_expression\ *\ int\_expression$\\
Rule 26 : $int\_expression \rightarrow int\_expression\ /\ int\_expression$\\
Rule 27 : $bool\_expression \rightarrow int\_expression\ >\ int\_expression$\\
Rule 28 : $bool\_expression \rightarrow int\_expression\ <\ int\_expression$\\
Rule 29 : $bool\_expression \rightarrow !\ bool\_expression$\\
Rule 30 : $bool\_expression \rightarrow int\_expression\ NEQ\ int\_expression$\\
Rule 31 : $bool\_expression \rightarrow bool\_expression\ OR\ bool\_expression$\\
Rule 32 : $bool\_expression \rightarrow bool\_expression\ AND\ bool\_expression$\\
Rule 33 : $bool\_expression \rightarrow TRUE$\\
Rule 34 : $bool\_expression \rightarrow FALSE$\\
Rule 35 : $expression \rightarrow IF\ bool\_expression\ DO\ expression\_list\ ENDIF$\\
\end{flushleft}

Dans notre parser qui reprend cette grammaire nous avons, à chaque règle trouvée, créé un tuple qui décrit l'opération en question. Ce tuple sera utilisé par notre interpréteur. \\


\textbf{Exemple :}\\
Pour la règle 18, on crée un tuple qui contient 
\begin{verbatim}
('FOR',p1,p2,p3)
\end{verbatim}
où \\p1 : variable qui va évoluer,\\ p2 : variable sur laquelle on va itérer,\\ p3 : instructions à réaliser. 

\section{For, If et explications}
\subsection{Explications}
Nous avons créé une variable globale dans notre programme qui contient des informations sur l'environnement des variables.
Cet environnement peut être enrichi en connaissance comme lorsque un ID est assigné. Alors on ajoute l'ID et sa valeur à la variable globale, ou on peut également retirer des informations lorsqu'une variable est initialisée dans une boucle et ne peut plus être appelée ensuite.
\subsection{For}
Comme décrit dans la section précédente, l'interpréteur va recevoir un tuple contenant plusieurs éléments d'information sur
la boucle For.

L'interpréteur va d'abord aller chercher le contenu de la variable cible.

Ensuite pour chaque élément $node$ de cette variable, une variable accessible par les instructions de la boucle va être assignée avec la valeur de $node$.
Une fois la valeur assignée et ajoutée à la variable globale d'environnement, on va exécuter le corps de la boucle, ainsi de suite jusqu'à ce que chaque élément $node$ soit parcouru.
\begin{lstlisting}
if (head == 'FOR'):
	acc = ""
	if env[t[1]] != None:
		copy_var = env[t[1]]
	l = interpret(t[2])
	for node in l:
		env[t[1]] = node
		acc += interpret(t[3])
	env[t[1]] = copy_var
	return acc
\end{lstlisting}
\subsection{If}
Le If est réalisé comme toutes les autres opérations à l'aide de la description contenue dans un tuple.
L'interpréteur va évaluer la condition si celle-ci est vérifié va interpréter l'intérieur du if sinon va retourner une chaine vide car l'entièreté du programme est une concaténation de chaines de caractères.
\begin{lstlisting}

if (head == 'IF'):
	if(interpret(t[1])):
		return interpret(t[2])
	else:
		return ""
\end{lstlisting}


\section{Problèmes rencontrés}
\subsection{Modification de grammaire}
Nous avons essayer de construire petit à petit notre programme en regardant à ce que chaque parties implémentées soient fonctionnelles avant de passer à la suite.
Mais dès lors que l'on veut changer la grammaire, et qu'un bug apparait, on ne peut pas directement déterminer de quelle partie du compilateur le problème provient.
Dans notre projet on a par exemple eu une regex qui était légèrement mal construite mais qui fonctionnait correctement avec notre lexer et notre grammaire, lorsque l'on a voulu faire évoluer notre grammaire des erreurs sont apparues mais ne provenait pas de la nouvelle description de la grammaire.
\subsection{Analyse du fichier de donnée}
Le fichier de donnée n'est censé que contenir des assignations de variable aucune instruction d'écriture ne doit apparaitre. Sur cette hypothèse donc nous avons concaténer ce fichier au programme dumbo, pour que les valeurs des variables puissent être connues du programme. Nous savons bien qu'un autre moyen était réalisable en parsant tout d'abord ce fichier de donnée a part et en parsant ensuite le programme dumbo, seulement notre application n'a pas été pensée de cette façon dès le début.
\end{document}